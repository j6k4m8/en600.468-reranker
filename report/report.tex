\documentclass{article}

\usepackage[english]{babel}
\usepackage[utf8x]{inputenc}
\usepackage{amsmath}
\usepackage{graphicx}

\title{Rerank}
\author{Jordan Matelsky}

\begin{document}
\maketitle

\section{Introduction}

\subsection{Runner.py}
In order to improve the reliability and reproducibility of the testing scheme that I use, I wrote a runner (runner.py) to make testing as simple as running a single line from the terminal.

\subsection{Testing}
Test a Python file by running \texttt{./runner [filename] [optional csv args]}. For instance, to test the baseline that was provided, type the following into the terminal: \texttt{./runner rerank-baseline}. To test the provided oracle, run \texttt{./runner oracle}.

This prints two lines: a tuple of results, in the order of $(oracle, tested, time)$, and a timestamp of how long the test took to run (not including the oracle).

\section{Improvement Strategy}
To improve over the baseline reranker, I followed several methods to improve upon the low baseline set by the BLEU computation on \texttt{rerank}.

\subsection{Non-ASCII Penalty}
This function, implemented in \texttt{Modes.DEFAULT\_CHECK\_ASCII}, deduces a penalty from the score of a hypothesis every time a word is encountered for which an ASCII-decode is impossible. Naturally, this fails in languages that use the Roman alphabet, but for this example (Cyrillic alphabet), I am able to improve BLEU using a penalty of 20 per non-ASCII (and thus, presumably non-translated) word.

\subsubsection{Advantages}
This prevents sentences that have many untranslated words from being considered `best'. It also seems to generally \textit{improve} BLEU — in my preliminary tests, I did not see cases where this reduced efficacy of the overall ranking.

This is also a very quick check to perform, and it likely serves as a good ``first-line of defense'' when considering hypotheses.

\subsubsection{Disadvantages}
This does not necessarily always give the best hypothesis. Consider a case like the following, where \# indicates an ASCII failure: \\


\textit{He is a conference last Thursday.} \\

\textit{\# went to the conference last Thursday.} \\

The second sentence is clearly superior in terms of conveying meaning, but the first sentence would be rated `better' by this algorithm alone. Thus, it is likely that we need to add the insight from other algorithms rather than run this one alone.

\subsubsection{Results}
This algorithm alone performs only marginally better than the standalone baseline. While the baseline scores $0.273509457562$, ASCII-checking scores $\approx0.003$ higher, $0.27\textbf{5}351661881$.

\begin{figure}[ht]
\caption{Naive speeds when running WORDNET (red) compared with dynamic version (blue), which saves the encountered words to a local dictionary for faster lookup.}
\label{fig:dynamic}
\centering
\includegraphics[width=0.5\textwidth]{figure_2}
\end{figure}

\end{document}
