\documentclass{article}

\usepackage[english]{babel}
\usepackage[utf8x]{inputenc}
\usepackage{amsmath}
\usepackage{graphicx}

\title{Rerank}
\author{Jordan Matelsky}

\begin{document}
\maketitle

\section{Introduction}

\subsection{Runner.py}
In order to improve the reliability and reproducibility of the testing scheme that I use, I wrote a runner (runner.py) to make testing as simple as running a single line from the terminal.

\subsection{Testing}
Test a Python file by running \texttt{./runner [filename] [optional csv args]}. For instance, to test the baseline that was provided, type the following into the terminal: \texttt{./runner rerank-baseline}. To test the provided oracle, run \texttt{./runner oracle}.

This prints two lines: a tuple of results, in the order of $(oracle, tested, time)$, and a timestamp of how long the test took to run (not including the oracle).

\section{Improvement Strategy}
To improve over the baseline reranker, I followed several methods to improve upon the low baseline set by the BLEU computation on \texttt{rerank}.

\subsection{Non-ASCII Penalty}
This function, implemented in \texttt{Modes.DEFAULT\_CHECK\_ASCII}, deduces a penalty from the score of a hypothesis every time a word is encountered for which an ASCII-decode is impossible. Naturally, this fails in languages that use the Roman alphabet, but for this example (Cyrillic alphabet), I am able to improve BLEU using a penalty of 20 per non-ASCII (and thus, presumably non-translated) word.

\subsubsection{Advantages}
This prevents sentences that have many untranslated words from being considered `best'. It also seems to generally \textit{improve} BLEU — in my preliminary tests, I did not see cases where this reduced efficacy of the overall ranking.

This is also a very quick check to perform, and it likely serves as a good ``first-line of defense'' when considering hypotheses.

\subsubsection{Disadvantages}
This does not necessarily always give the best hypothesis. Consider a case like the following, where \# indicates an ASCII failure: \\

\textit{He is a conference last Thursday.} \\

\textit{\# went to the conference last Thursday.} \\

The second sentence is clearly superior in terms of conveying meaning, but the first sentence would be rated `better' by this algorithm alone. Thus, it is likely that we need to add the insight from other algorithms rather than run this one alone.

\subsubsection{Results}
This algorithm alone performs only marginally better than the standalone baseline. While the baseline scores $27.35$, ASCII-checking scores $\approx0.003$ higher, $27.\textbf{5}4$.

When the \texttt{CHECK\_ASCII} algorithm is run on its own, it performs slightly worse than baseline, at $26.3$.

\subsection{Word Count}
Next, I implemented a word-count feature, as this had been widely regarded as a useful measure in previous work \cite{upennstudent,hmm11}. Because BLEU scores fundamentally correlate with sentence-length, it is important that we consider this when scoring sentences in order to prevent extreme-length sentences from being weighted improperly.

\subsubsection{Advantages}
Again, this is a very quick implementation, but it is, with regards to its speed, disproportionately (very!) useful. (See \textit{Results} for numerical results.)

\subsubsection{Disadvantages}
This approach naturally fails in cases where sentence-length differs dramatically. Consider languages such as German where compound-words reduce the length of a sentence in word-count, but maintain meaning. BLEU is a useful metric for sentence structure in some languages, but a more general algorithm should be used for a general translation scoring mechanism.

\subsubsection{Results}
When run alone, word count is (not unexpectedly) useless, scoring well below baseline. However, when run in conjunction with the default scoring system (sum), it performs notably better than baseline, scoring $28.7$ (baseline = $27.35$).

When all three thus-far explored algorithms are run together (Default baseline, ASCII, and word-count), BLEU-score reaches $28.8$.

% \begin{figure}[ht]
% \caption{Naive speeds when running WORDNET (red) compared with dynamic version (blue), which saves the encountered words to a local dictionary for faster lookup.}
% \label{fig:dynamic}
% \centering
% \includegraphics[width=0.5\textwidth]{figure_2}
% \end{figure}

\medskip

\begin{thebibliography}{9}
\bibitem{upennstudent}
UPenn Student (Name not provided).
\textit{CIS526 HW4 Report}.
University of Pennsylvania, 2015. \texttt{http://www.seas.upenn.edu/~cis526/reports/hw4/828463.pdf}

\bibitem{hmm11}
Hopkins, Mark, and Jonathan May. \textit{``Tuning as ranking.''}
Proceedings of the Conference on Empirical Methods in Natural Language Processing. Association for Computational Linguistics, 2011.

\end{thebibliography}

\end{document}
